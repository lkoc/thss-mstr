
% Thesis Abstract -----------------------------------------------------


%\begin{abstractslong}    %uncommenting this line, gives a different abstract heading
\begin{abstracts}        %this creates the heading for the abstract page

El problema de calcular el viento sobre terrenos complejos ha recibido gran atención en los últimos años debido al incremento de la generación eléctrica con turbinas eólicas en diversas partes del mundo. La simulación numérica mediante la discretización de las ecuaciones de Navier-Stokes agiliza el desarrollo de los proyectos eólicos mediante una mejor estimación del recurso energético que hace viable a este tipo de proyectos. Un balance entre esfuerzo computacional y precisión de los resultados se requiere; así las ecuaciones de promedios de Reynold (RANS) y la caracterización de la  turbulencia mediante modelos de transporte de la energía cinética turbulenta y el ratio de disipación de esta energía (modelo $\kappa$-$\epsilon$ ) en estado estable han sido usados en el presente trabajo. Así mismo, la interacción entre el flujo turbulento del aire, los terrenos con topografía compleja y la capa límite atmosférica son analizadas, adicionalmente en estado dinámico utilizando el método de lattice-Boltzman (LBM).
\\
\\
El presente trabajo presenta una metodología para calcular la velocidad de viento, intensidad de turbulencia y otras magnitudes relevantes en las áreas extensas sobre la que se construyen parques eólicos. El calculo incluye valores a diferentes altitudes; así como, la interacción con las turbinas eólicas de modos que incluyen las pérdidas por estelas. Se utilizan los métodos RANS $k$-$\epsilon$ (usando el código OpenFoam) y Lattice-Boltzman (usando el código OpenLB). Los resultados son validados mediante la resolución de problemas canónicos y comparaciones con valores medidos en el campo obteniendo resultados consistentes.


%\blindtext

\end{abstracts}
%\end{abstractlongs}


% ----------------------------------------------------------------------